\documentclass{resume}
\usepackage{zh_CN-Adobefonts_external}
\usepackage{linespacing_fix}
\usepackage{cite}
\begin{document}
\pagenumbering{gobble}


%***"%"后面的所有内容是注释而非代码,不会输出到最后的PDF中
%***使用本模板,只需要参照输出的PDF,在本文档的相应位置做简单替换即可
%***修改之后,输出更新后的PDF,只需要点击Overleaf中的“Recompile”按钮即可
%**********************************姓名********************************************
\name{梁劲舟}
%**********************************联系信息****************************************
%第一个括号里写手机号,第二个写邮箱
\contactInfo{(+86) 18176608865}{dok4ever@qq.com}
%**********************************其他信息****************************************
%在大括号内填写其他信息,最多填写4个,但是如果选择不填信息,
%那么大括号必须空着不写,而不能删除大括号。
%\otherInfo后面的四个大括号里的所有信息都会在一行输出
%如果想要写两行,那就用两次这个指令(\otherInfo{}{}{}{})即可
\otherInfo{性别:男}{籍贯:广西}{}{}
%*********************************照片**********************************************
%照片需要放到images文件夹下,名字必须是you.jpg,如果不需要照片可以不添加此行命令
%0.15的意思是,照片的宽度是页面宽度的0.15倍,调整大小,避免遮挡文字
\yourphoto{0.12}
%**********************************正文**********************************************


%***大标题,下面有横线做分割
%***一般的标题有:教育背景,实习(项目)经历,工作经历,自我评价,求职意向,等等
\section{教育背景}


%***********一行子标题**************
%***第一个大括号里的内容向左对齐,第二个大括号里的内容向右对齐
%***\textbf{}括号里的字是粗体,\textit{}括号里的字是斜体
\datedsubsection{\textbf{桂林电子科技大学},数学,\textit{理学硕士}}{2020.09 - 2023.06}


%***********列举*********************
%***可添加多个\item,得到多个列举项,类似的也可以用\textbf{}、\textit{}做强调
\begin{itemize} [parsep=1ex]
  \item \textbf{主修课程:}泛函分析,抽象代数,最优化理论与算法,矩阵论等。
  \item 2021年全国大学生数学竞赛数学类专业三等奖
  \item 2023年全国大学生数学竞赛数学类专业二等奖
\end{itemize}


\datedsubsection{\textbf{西安邮电大学},数学,\textit{理学学士}}{2016.09 - 2020.06}
\begin{itemize} [parsep=1ex]
  \item \textbf{主修课程:}数学分析,高等代数,概率论,梳理统计等
  \item 2018年高教杯全国大学生数学竞赛二等奖
\end{itemize}
%%%%%%%%%%%%%%%%%%%%

\section{校园经历}

\datedsubsection{\textbf{桂林电子科技大学第二十三届研究生支教团}}{2021}
\begin{itemize}
  \item 前往桂林市全州县凤凰镇进行支教志愿服务, 任七年级五班班主任和数学教师一职. 帮助学生提升成绩,同时积极带领学生开展校内外活动,促进学生多方面发展。
\end{itemize}

\datedsubsection{\textbf{院学生会\, 干事}}{2020.09-2023.04}
\begin{itemize}
  \item 与其他干事合作, 成功举办多次大型活动, 如迎新晚会, 青年座谈会, 校运动会等.
\end{itemize}


%%%%%%%%%%%%%%%%%%%%%
\section{合作项目}

\datedsubsection{\textbf{求解单调包含问题的快速算法}}{2021.03-2021.06}
\begin{itemize}[parsep=0.5ex]
  % \item 在广西数学研究中心,作为核心成员参与了一项关于图像处理算法的项目。该项目要求在严格的实验条件下,分析该算法的性能和适用性。
  \item 在广西数学研究中心,作为核心成员参与了一项关于图像处理算法的项目。负责对小组工作的监督和指导,工作内容包括对工作进度、每周组会进行落实,并定期向导师和实验室主管汇报进度。最终研究成果发表在国际知名学术期刊上,受到了同行的认可。
  % \item 负责领导三人小组,负责对小组工作的监督和指导,工作内容包括对工作进度、每周组会进行落实,并定期向导师和实验室主管汇报进度。还要对大量的数据进行处理筛选,确保实验数据的准确性和可靠性,为整个项目提供有力支持。
  % \item 最终成功完成算法性能的初步分析,为整个问题的解决提供了有力支持。同时,我们的研究成果也在国际知名学术期刊上发表,受到了同行的高度评价。
\end{itemize}

\datedsubsection{\textbf{硕士毕业论文}}{2022.03 - 2023.05}
\begin{itemize}[parsep=0.5ex]
  % \item 在攻读硕士学位期间,选择最优化领域作为研究方向,专注于解决单调变分问题,提出有效的解决方案,并通过实证分析验证其可行性和有效性。这个领域的研究面临着数据获取难、分析方法复杂等挑战。
  \item 对该问题提出有效的解决方案,并通过实证分析验证其可行性和有效性。为了解决问题,查阅相关领域的文献,确定问题背景和理论基础; 对实验得到的数据进行处理和分析,得到初步的结果。最后完成硕士论文的撰写,得到导师和评审专家的认可。
  % \item 为了解决问题,查阅了相关领域的文献,确定了问题背景和理论基础; 通过各种渠道获得大量实验数据; 根据研究问题的特点,进行了详细的设计和规划; 对实验得到的数据进行处理和分析,得到初步的结果,形成了论文的大纲和初稿。
  % \item 最后成功完成硕士论文的撰写。提出有效的解决方案。论文得到导师和评审专家的好评。
\end{itemize}

\section{技能证书}

\begin{itemize}
  \item \textbf{普通话二级甲等}
  \item \textbf{大学英语四六级, 具有良好的听说读写能力, 能够快速阅读英文资料和书籍}
  \item \textbf{全国计算机二级考试}
\end{itemize}

\section{获奖情况}

\datedsubsection{\textbf{2020年桂林电子科技大学研究生奖学金三等奖}}{2020.09}

\datedsubsection{\textbf{2021年桂林电子科技大学研究生奖学金三等奖}}{2021.10}

\datedsubsection{\textbf{2022年桂林电子科技大学研究生奖学金二等奖}}{2022.10}

\datedsubsection{\textbf{2023年桂林电子科技大学研究生优秀论文}}{2023.06}
%%%%%%%%%%%%%%%%%%%




\end{document}
